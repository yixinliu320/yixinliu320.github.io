\documentclass[12pt]{article}
%%% General Formatting %%%%
\usepackage{amsfonts, amsmath, amssymb, bm} %Math fonts and symbols
%\usepackage{times} % using Times New Roman
\usepackage[colorlinks=true]{hyperref}%
\usepackage {graphicx,  float} % graphics commands
\usepackage[margin=1in]{geometry} % sets page layout
\usepackage[singlespacing]{setspace}% allows toggling of double/single-spacing
\usepackage{verbatim}% defines environment for un-evaluated code
\usepackage{breqn} %mutiple line math function

% Format section title 
\usepackage{titlesec} %his is a package for title
\titleformat*{\section}{\normalfont\fontsize{14}{14}\bfseries} %bfseries means bold font
\titleformat*{\subsection}{\normalfont\fontsize{12}{12}\bfseries}
\titleformat*{\subsubsection}{\normalfont\fontsize{12}{12}\it} % \it means italics


\begin{document}
\date{\today}
\title{Introduction to \LaTeX}
\maketitle


\begin{center}
\author{\large Yixin Liu}
\end{center}

\section{Introduction} % you can change the section label in title format part, or just text without label use: section*{}
First rule: You can find everything about \LaTeX \ on Google. \\ 
Make text \textbf{bold}. \\
Make text \textit{italics}. \par
Make text \underline{underline}. 
 
\subsection{Different font size}
{\Huge Hello} {\LARGE Hello} {\Large Hello} {\large Hello} {\normalsize Hello} {\small Hello} {\footnotesize Hello} {\scriptsize Hello} {\tiny Hello}\\

\subsubsection{Different font families}
\textrm{Kobe Bryant} \\ 
\textsf{Kobe Bryant} \\ 
\texttt{Kobe Bryant} \\

\section{Mathematical expressions}
Microsoft Word is \textit{terrible} at math! \\ 
You can insert math in text $\lim\limits_{x \to \infty} \exp(-x) = 0$ or put it individually as below. \\

\begin{equation}
k_{n+1} = n^2 + k_n^2 - k_{n-1}
\end{equation}

Even with multiple lines:\\

\begin{dmath}
  Q(\lambda,\hat{\lambda}) = -\frac{1}{2} P{(O \mid \lambda )} \sum_s \sum_m \sum_t \gamma_m^{(s)} (t) \left( n \log(2 \pi ) + \log \left| C_m^{(s)} \right| + \left( \mathbf{o}_t - \hat{\mu}_m^{(s)} \right) ^T C_m^{(s)-1} \left(\mathbf{o}_t - \hat{\mu}_m^{(s)}\right) \right)
\end{dmath}

\section{Table}
Table function is based on tabular, which allow us to customize every element in it. 

\begin{center}
\begin{tabular}{ l r c c}
 A & B & cell3 & hi\\
 cell4 & cell5 & cell6 & a\\  
 cell7 & cell8 & cell9  & a  
\end{tabular}
\end{center}

\begin{center}
\begin{tabular}{ |c|c|c| }
\hline
 cell1 & cell2 & cell3 \\ 
 \hline
 cell4 & cell5 & cell6 \\ 
 \hline
 cell7 & cell8 & cell9 \\
 \hline 
\end{tabular}
\end{center}

Now, see some complicated examples: 
\bigskip

\begin{tabular}{ p{3.5cm} p{10cm} }
2017-2022 & \textbf{Florida State University, Askew School of Public Administration} \\
   & PhD in Public Administration and Policy \\
   & {\it Committee}: XXX, XXX, XXX, XXX\\
   & \\
2014-2016 & \textbf{Rutgers University, Bloustein School of Planning and Public Policy} \\
   &Master in Public Policy \\
   & {\it Advisor}: XXX \\
   & \\
2009-2013 & \textbf{Hong Kong Baptist University} \\
   &BA in Communication

\end{tabular}

\begin{table}[h!]
\centering
\footnotesize
\caption{Descriptive Statistics} 
\label{tab:descriptive}
\renewcommand{\arraystretch}{1} % change the height of table
\begin{tabular}{lrrrrr}
  \hline\hline \\[-2.5ex]
 & Obs.& Mean & SD & Min & Max  \\ 
  \hline \\[-2.5ex]
  Crime number &651& 710.78 & 3193.29 & 0.00 & 28429.00  \\ 
  Turnover ratio &651 & 7.71 & 14.94 & 0.00 & 100.00  \\ 
  Total collaborators & 651&  18.35 & 9.44 & 0.00 & 44.00  \\ 
  New collaborators & 651 & 1.11 & 2.71 & 0.00 & 20.00  \\ 
  Terminated Collaborators &651 & 1.02 & 2.74 & 0.00 & 21.00  \\ 
  Population &651  & 19806.34 & 63691.70 & 342.00 & 546877.00 \\ 
  Median household income &651 & 49154.21 & 7031.21 & 32292.00 & 79549.00  \\ 
  Unemployment rate & 651 & 3.15 & 0.75 & 1.90 & 8.70  \\ 
  Ethnic diversity index & 651 & 10.30 & 10.11 & 0.54 & 52.72  \\ 
  Officers coverage & 651& 1.68 & 0.96 & 0.00 & 8.80  \\ 
  Election & 651& 0.57 & 0.50 & 0.00 & 1.00  \\ 
  Mobility & 651& 1342.19 & 4102.07 & 8.00 & 32385.00  \\ 
  Property tax & 651 & 6300551.30 & 13127090.21 & 427953.00 & 120208368.00  \\ 
  \hline\hline 
\end{tabular}
\end{table}

\clearpage
\section{Figure}
\LaTeX \ allows EPS format figure. 


%%%insert your own pic
\begin{figure}[!htbp]
	\centering
	\includegraphics[height=7cm]{ate}
	\caption{Average Treatment Effect}
	\label{ate}
\end{figure}

%%%insert your own pic
\begin{figure}[!htbp]
	\centering
	\includegraphics[height=7cm]{gap}
	\caption{PI Gaps in The Real Treated Unit and Placebo PI Gaps in Control Units}
	\label{gap}
\end{figure}


\section{Itemize}
\subsection{Basic}
\begin{itemize}
  \item One entry in the list
  \item Another entry in the list
\end{itemize}

\subsection{Use number label}
\begin{enumerate}
  \item The labels consists of sequential numbers.
  \item The numbers starts at 1 with every call to the enumerate environment.
\end{enumerate}


\subsection{Nest list}
\begin{enumerate}
   \item The labels consists of sequential numbers.
   \begin{itemize}
     \item The individual entries are indicated with a black dot, a so-called bullet.
     \item The text in the entries may be of any length.
   \end{itemize}
   \item The numbers starts at 1 with every call to the enumerate environment.
\end{enumerate}

\section{Assignment}
Now, try to use \LaTeX \ to polish your resume. And add \LaTeX \ into your professional skill section. 

\end{document}

